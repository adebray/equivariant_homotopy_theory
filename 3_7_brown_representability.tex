\begin{quote}\textit{
	``I thought it was obvious, but it's obviously not obvious.''
}\end{quote}
Today, we're going to finish the additive correspondence between $G$-spectra and $\RO(G)$-graded cohomology
theories, through Neeman's proof of Brown representability. There will be no class Thursday, so we'll resume after
spring break with some computations of $\RO(G)$-graded cohomology theories, including ordinary cohomology and
hopefully also twisted $K$-theory.

Anyways, recall that we defined $\RO(G)$-graded cohomology theories $\set{E_G^V(X)}$ to be Mackey functors graded
on the category $\cat{RO}(G;U)$, where $U$ is the ambient universe, together with isomorphisms $\sigma^W\colon
E_G^V(X)\cong E_G^{V\oplus W}(\Sigma^W X)$, together with some additional data.

Before we go on, though, we should talk a little about $\RO(G)$.
\begin{defn}
Let $G$ be a group and $R$ be a ring. Then, the \term{representation ring} of $G$ over $R$ is the Grothendieck
construction applied to the category of finite-rank $G$-representations over $R$, i.e.\ the ring of isomorphism
classes of finite-rank $G$-representations over $R$, where addition is direct sum and multiplication is tensor
product. By \emph{the} representation ring, we'll mean $\RO(G)\coloneqq\R(G)$, the ring of real representations.
\end{defn}
We can think of $\R(G)$ and $\C(G)$ as a ring of characters: given a representation $V$, its \term{character} is
the map $\chi_V\colon G\to\R$ (or to $\C$) where $\chi_V(g) = \tr(g\cdot)$. Since $\chi_{V\oplus W} = \chi_V +
\chi_W$, $\chi_{V\otimes W} = \chi_V\cdot\chi_W$, and for irreducible representations, $\chi_V = \chi_W$ iff
$V\cong W$, then the characters detect isomorphism classes, so $\R(G)$ and $\C(G)$ can be thought of as the ring of
functions $G\to\R$ (or $\C$) that are characters.
\begin{comp}{ex}{enumerate}
	\item Compute $\C(G)$ for $G = C_p$, and then for $C_m$ where $m$ isn't prime.
	\item Harder: compute $\R(G)$ for the same $G$.
\end{comp}
One of the important connections in ordinary stable homotopy theory is that cohomology theories are closely related
to spectra: every spectrum determines a cohomology theory, and up to a very small ambiguity, a cohomology theory is
represented by a spectrum.
\begin{defn}
Let $E$ be a $G$-spectrum. Then, we define \term{$E$-cohomology} to be the functor $E_G^V(X)\coloneqq [S^V\wedge X,
E]_G$.
\end{defn}
This is an $\RO(G)$-graded cohomology theory, but this is something to check. If you want to restrict your
universe, there are some nuances that have to be overcome.

If you put in finite $G$-sets (basically points), what you get is a Mackey functor: considering maps out of $X$
ensures the transfer maps have the correct variance.

That was the easy direction: the other direction requires Brown representability. This is originally due to
\cite{Brown}, and we'll give Neeman's interpretation~\cite{Neeman}. It works in any triangulated category, and is
very close to the small object argument, which already makes it a good thing.

Fix a triangulated category $\fC$ that has small coproducts. The stable homotopy categories $\Ho(\Spc)$ and
$\Ho(\Spc^G)$ are the examples to keep in mind.
\begin{defn}
An $x\in\fC$ is \term{compact} if for all countable coproducts over $y_i\in\fC$,
\[\Map_\fC\paren{x, \coprod_i y_i} \cong \coprod_i \Map_\fC(x, y_i).\]
\end{defn}
This is not the usual definition of compactness in category theory, which uses filtered colimits, but in the stable
setting these are the same as coproducts, motivating our definition. For example, this definition does not
characterize compact topological spaces.
\begin{comp}{defn}{itemize}
	\item A \term{generating set} of a category $\fC$ is a set $T\subseteq\operatorname{ob}(\fC)$ that
	\term{detects zero}, i.e.\ for all $x\in\fC$, $x = 0$ iff $\Map_\fC(z,x) = 0$ for all $z\in T$. If $\fC$ is
	triangulated, we additionally require $T$ to be closed under shift.
	\item $\fC$ is \term{compactly generated} if it has a generating set consisting of compact objects.
\end{comp}
We introduce these to skate around set-theoretic issues: at some point, we'd like to take a coproduct over all
objects in the category, but that's too large. Instead, taking the coproduct over all generators will have the same
power, and is actually well-defined.

For example, dualizable spectra (resp.\ dualizable $G$-spectra) are a compact generating set for $\Ho(\Spc)$
(resp.\ $\Ho(\Spc^G)$).
\begin{thm}[Brown representability~\cite{Brown, Neeman}]
Let $\fC$ be a compactly generated triangulated category and $H\colon\fC\op\to\Ab$ be a functor such that
\begin{enumerate}
	\item
	\[H\paren{\coprod_i X_i}\cong\prod_i H(X_i)\]
	and
	\item $H$ sends exact triangles $X\to Y\to Z\to X[1]$ to long exact sequences
	\[\xymatrix{
		H(X)\ar[r] & H(Y)\ar[r] & H(Z)\ar[r] & H(X[1])\ar[r] & H(Y[1])\ar[r] & H(Z[1])\ar[r] & H(X[2])\ar[r] &
		\dotsb
	}\]
\end{enumerate}
Then, $H$ is \term{representable}, i.e.\ there's an $X\in\fC$ and a natural isomorphism $\Hom_\fC(\bl,X)\to H$.
\end{thm}
\begin{proof}
We're going to build $X$ inductively. Fix a generating set $T$ for $\fC$ of compact objects.

In the base case, let
\[U_0\coloneqq \coprod_{x\in T} H(x)\qquad\text{and}\qquad X_0 \coloneqq \coprod_{\substack{(\alpha,t)\in
U_0\\\alpha\in H(t)}} t.\]
Thus,
\[H(X_0) = H\paren{\coprod_{(\alpha, t)\in U_0} t}\cong\prod_{(\alpha, t)\in U_0} H(t),\]
and in particular there is a distinguished element $\alpha_0\in H(X_0)$ which is $\alpha$ at the $(\alpha,t)$
factor. By the Yoneda lemma, $\alpha_0$ specifies a natural transformation $\theta_0\colon \Map_\fC(\bl,X_0)\to H$,
and by construction, $\theta_0$ is surjective for each $t\in T$.

Now we induct: assume we have $X_i$ and $\alpha_i$ specifying a natural transformation $\theta_i\colon
\Map_\fC(\bl,X_i)\to H$. Then, define
\[U_{i+1}\coloneqq \coprod_{t\in T} \ker(\theta_i(t)\colon\Map_\fC(t,X_i)\to H(t)).\qquad\text{and}\qquad
K_{i+1}\coloneqq \coprod_{(f,t)\in U_{i+1}} t.\]
There's a natural map $K_{i+1}\to X_i$ which applies $f$; let $X_{i+1}$ be its cofiber. Applying $H$, this
produces a map
\[\xymatrix{
	H(X_{i+1})\ar[r] & H(X_i)\ar[r] & H(K_{i+1}),
}\]
and by construction, $\alpha_i\mapsto 0$, so by exactness, we can lift to $\alpha_{i+1}\in H(X_{i+1})$. This
$\alpha_{i+1}$ specifies a natural transformation $\theta_{i+1}\colon \Map_\fC(\bl,X_{i+1})\to H$, and the
following diagram commutes:
\[\xymatrix{
	\Map_\fC(\bl,X_i)\ar[r]^-{\theta_i}\ar[d] & H\\
	\Map_\fC(\bl,X_{i+1}).\ar[ur]_{\theta_{i+1}}
}\]
Thus we have a tower $\N\to\fC$ sending $i\mapsto X_i$, and we can define
\[X\coloneqq \hocolim_i X_i,\]
which we'll show represents $H$. The \term{homotopy colimit} in a triangulated category is \TODO.
\begin{rem}
Though triangulated categories are a good language to know, as they're historically interesting and often useful,
they have drawbacks: the octahedral axiom is awkward, for example, and sometimes the triangulated structure works
against you.

For this reason, it can be useful to remember that triangulated categories arise as the homotopy categories of
stable $\infty$-categories, where there's additional versatility simplifying some arguments. For this reason, we
use words such as ``cofiber'' and ``homotopy colimit,'' because they're secretly the same thing. % TODO cofiber is
% assoc to any map...
\end{rem}
\begin{ex}
The homotopy colimit is also called the \term{telescope}. Relate this to the usual definition of a telescope.
\end{ex}
Anyways, we'll construct a natural transformation $\theta\colon\Map_\fC(\bl,X)\to H$ as follows. Since $H$ sends
cofiber sequences to long exact sequences, applying it to
\[\xymatrix{
	\coprod_i X_i\ar[r] & \coprod_i X_i\ar[r] & X
}\]
produces a long exact sequence
\[\xymatrix{
	\dotsb\ar[r] & H(X)\ar[r] & \prod_i H(X_i)\ar[r] & \prod_i H(X_i)\ar[r] & \dotsb,
}\]
so in particular we can lift the $\alpha_i\in X_i$ to an $\alpha\in X$, which defines $\theta$ as before, and there
is a commutative diagram
\[\xymatrix{
	\Map_\fC(\bl,X_0)\ar[r]^-{\theta_0}\ar[d] & H.\\
	\Map_\fC(\bl,X)\ar[ur]_{\theta}
}\]
In particular, $\theta$ is surjective on $T$ (namely, $\Map(t,X)\to H(t)$ is surjective for $t\in T$).

To see that $\theta$ is injective on $T$, we use compactness. Let $f\in\Map_\fC(t,X)$ be such that $\theta(f) = 0$;
we wish to show that $f = 0$. Since $t$ is compact, $f$ lies in some $\Map_\fC(t,X_i)$, i.e.\
$f\in\ker(\Map(t,X_i)\to H(t))$. Therefore, $f\in U_{i+1}$, so it's killed in $X_{i+1}$, and thus is $0$ in the
colimit. Therefore $\theta$ is injective on $T$, hence and isomorphism.

There exists a largest full triangulated subcategory $\widetilde\fC$ of $\fC$ that's closed under small coproducts
and such that $\theta|_{\widetilde\fC}$ is a natural isomorphism; we'll show that $\widetilde\fC =
\fC$.\footnote{Dissecting what ``largest full triangulated subcategory'' means requires a little care, but such a
$\widetilde\fC$ exists.}

Running the whole argument again,\footnote{It's important that we use the same generating set $T$ for this.} we
obtain a $Z\in\widetilde\fC$ and a natural transformation $\widetilde\theta\colon\Map_{\widetilde\fC}(\bl,Z)\to H$.
We know $\Map_{\widetilde\fC}$ and $\Map_\fC$ are isomorphic, and proved that $\theta$ and $\widetilde\theta$ are
isomorphisms on the generating set, we can consider the cofiber of
\[\Map_\fC(\bl,Z)\longrightarrow\Map_\fC(\bl,X).\]
By the Yoneda lemma, such natural transformations are naturally identified with $\Map_\fC(Z,X)$, and you can
compute the cofiber of $Z\to X$ in terms of these maps. This is $0$, so $Z\cong X$.

\TODO: from here, we need to conclude why $\widetilde\fC = \fC$. This ought to be true, but we got stuck.
\end{proof}
Next time, we'll discuss what Brown representability means for $G$-spectra, and use it to define Eilenberg-Mac Lane
spectra as the spectra representing certain cohomology theories. One fun fact about $G$-spectra is that every
$\Z$-graded cohomology theory uniquely extends to an $\RO(G)$-graded cohomology theory by taking shifts, but this
is not true for $G$-spaces!
