This section was not part of the class: it was delivered by Richard Wong as part of a mini-course on spectral
sequences in equivariant stable homotopy theory. See
\url{https://www.ma.utexas.edu/users/richard.wong/2017/Resources.html} for more information.

We'll start with the Bousfield-Kan spectral sequence (BKSS). One good reference for this is Guillou's
notes~\cite{Guillou}, and Hans Baues~\cite{Baues} set it up in a general model category.

We'll work in $\sSet$, so that everything is connective. Consider a tower of fibrations
\begin{equation}
\label{fibtower}
\xymatrix{
	\dotsb\ar[r] & Y_s\ar[r]^{p_s} & Y_{s-1}\ar[r]^{p_{s-1}} &Y_{s-2}\ar[r] & \dotsb 
}
\end{equation}
for $s\ge 0$, and let $Y\coloneqq \varprojlim Y_s$. Let $F_s$ be the fiber of $p_s$.
\begin{thm}[Bousfield-Kan~\cite{BK72}]
In this situation, there is a spectral sequence, called the \term{Bousfield-Kan spectral sequence}, with signature
\[E_1^{s,t} = \pi_{t-s}(F_s)\Longrightarrow \pi_{t-s}(Y).\]
\end{thm}
If everything here is connective (which is not always the case in other model categories, as in one of our
examples), this is first-quadrant. One common convention is to use the \term{Adams grading} $(t-s,s)$ instead of
$(s,t)$.

We can extend~\eqref{fibtower} into a diagram
\[\xymatrix{
	\dotsb\ar[r] & Y_s\ar[r]^{p_s} & Y_{s-1}\ar[r]^{p_{s-1}} &Y_{s-2}\ar[r] & \dotsb\\
	& F_s\ar[u]_{i_s} & F_{s-1}\ar[u]_{i_{s-1}} & F_{s-2},\ar[u]_{i_{s-2}}
}\]
and hence into an exact couple\index{exact couple}
\[\xymatrix{
	\dotsb\ar[r] & \pi_*(Y_{s+1})\ar[r] & \pi_*(Y_s)\ar[r]\ar[dl]_\delta & \pi_*(Y_{s-1})\ar[dl]_\delta\ar[r] &
	\dotsb\\
	& \pi_*(F_{s})\ar[u]^{i_{s*}} & \pi_*(F_{s-1})\ar[u]^(0.4){i_{(s-1)*}} &
	\pi_*(F_{s-2}),\ar[u]^(0.4){i_{(s-2)*}}
}\]
and the differentials are the compositions of the maps $\pi_*(F_s)\to\pi_*(Y_s)\to \pi_*(F_{s+1})$:
\[\xymatrix{
	\dotsb\ar[r] & \pi_*(Y_{s+1})\ar[r] & \pi_*(Y_s)\ar[r]\ar[dl]_\delta & \pi_*(Y_{s-1})\ar[dl]_\delta\ar[r] &
	\dotsb\\
	& \pi_*(F_{s})\ar[u]^{i_{s*}}\ar@[red][l] & \pi_*(F_{s-1}) \ar[u]^(0.4){i_{(s-1)*}}
	\ar@[red][l]^{\textcolor{red}{d_1}} & \pi_*(F_{s-2}). \ar[u]^(0.4){i_{(s-2)*}}
	\ar@[red][l]^{\textcolor{red}{d_1}} &\ar@[red][l]
}\]
Taking homology, we'll get a differential $d_2$ that jumps two steps to the left, then $d_3$ three steps to the
left, and so on. After you check that $\Im(d_r)\subset\ker(d_r)$, you can define $E_r^{s,s+1}\coloneqq
\ker(d_r)/\Im(d_r)$. Let $A_s\coloneqq\Im(\pi_0(Y_{s+r})\to\pi_0(Y_s))$, and let $Z_r^{s,s}\coloneqq
(i_s)^{-1}_*(A_s)$. Then, $E_{r+1}^{s,s} = Z_r^{s,s} / d_r(E_r^{s-r, s-r+1})$.
\begin{rem}
One important caveat is that for $i\le 2$, $\pi_i$ does not produce abelian groups, but rather groups or just sets!
This means that a few of the columns of this spectral sequence don't quite work, but the rest of it is normal, and
the degenerate columns can still be useful. This is an example of a \term{fringed spectral sequence}.
\end{rem}
Bousfield and Kan cared about this spectral sequence because it allowed them to write down a useful long exact
sequence, the \term[derived homotopy sequence]{$r^{\text{th}}$ derived homotopy sequence}: let
$\pi_iY^{(r)}\coloneqq \Im(\pi_i(Y_{n+r})\to\pi_i(Y_n))$; then, there's a long exact sequence
\[\xymatrix{
	\dotsb\ar[r] & \pi_{t-s-1}Y_{s-r-1}^{(r)}\ar[r] & E_{r+1}^{s,t}\ar[r] & \pi_{t-s}Y_s^{(r)}\ar[r]^\delta &
	\pi_{t-s}Y_{s-1}^{(r)}\ar[r] & E_{r+1}^{s+r, t+r+1}\ar[r] & \pi_{t-s+1}Y_s^{(r)}\ar[r] & \dotsb
}\]
You can do something like this in general given a spectral sequence, though you need to know how to obtain it from
the exact couple.
\begin{rem}
When $r = 0$, $E_1^{s,t} = \pi_{t-s}(F_s)$, and so the first derived homotopy sequence is the long exact sequence
of homotopy groups of a fibration.
\end{rem}
One nice application is to \term{Tot towers} (``Tot'' for totalization).
\begin{defn}
Let $X^\bullet$ be a cosimplicial object in $\sSet$. Then, its \term{totalization} is the complex
\[\Tot(X^\bullet) \coloneqq \sSet(\Delta^\bullet, X^\bullet),\]
i.e.
\[\Tot_n(X^\bullet) \coloneqq \sSet(\sk_n\Delta^\bullet, X^\bullet).\]
Here $(\sk_n\Delta^\bullet)^n\coloneqq \sk_n\Delta^m$.
\end{defn}
Then
\[\varprojlim\Tot_n(X^\bullet) = \Tot(X^\bullet),\]
reconciling the two definitions.
\begin{ex}
In the Reedy model structure, $\Tot_n(X^\bullet)\to\Tot_{n-1}(X^\bullet)$ is a fibration.\index{Reedy model
structure}
\end{ex}
Assuming this exercise, we can apply the Bousfield-Kan spectral sequence.

One place this pops up is that if $C,D\in\fC$ and $X_\bullet\to C$ is a simplicial resolution in a simplicial
category $\fC$,\footnote{Meaning that after geometrically realizing, there's an equivalence.} then
$\Hom_{\fC}(X_\bullet, D)$ is a cosimplicial object, and this spectral sequence can be used to compute
homotopically meaningful information about $\sSet(C, D)$.

We can use this formalism to derive the homotopy fixed point spectral sequence. Recall
(\cref{spectra_fixed_points}) that if $X$ is a $G$-spectrum, its homotopy fixed point spectrum\index{homotopy fixed
points!of a $G$-spectrum} is the nonequivariant spectrum $X^{hG}\coloneqq F((EG)_+, X)^G$, i.e.\ the spectrum
of $G$-equivariant maps $(EG)_+\to X$.\footnote{Notationally, this is the function spectrum of maps from
$\Sigma^\infty (EG)_+$ to $X$, or you can use the fact that spectra are cotensored over spaces.} The bar
construction\index{bar construction} gives us a simplicial resolution of $(EG)_+$, producing a cosimplicial object
that can be plugged into the Bousfield-Kan spectral sequence. Specifically, we write $EG = B^\bullet(G,G,*)$, add
a disjoint basepoint, and then take maps into $X$.
\begin{thm}
\label{HFPSS}
If $X$ is a $G$-spectrum, there's a spectral sequence, called the
\term{homotopy fixed point spectral sequence}, with signature
\[E_2^{p,q} = H^p(G, \pi_q(X))\Longrightarrow \pi_{q-p}(X^{hG}).\]
\end{thm}
\TODO: discuss the multiplicative structure.
\begin{exm}
The first example is really easy. Let $k$ be a field, and consider the Eilenberg-Mac Lane spectrum $Hk$. Let $G$
act trivially on $k$; we want to understand $\pi_*(Hk^{hG})$. The homotopy fixed-points spectral sequence is
particularly simple:\index{equivariant Eilenberg-Mac Lane spectrum}
\[E_2^{p,q} = H^p(G; \pi_q(Hk)) = \begin{cases}
	H^p(G; k), &q = 0\\
	0, &\text{otherwise.}
\end{cases}\]
Since this is a single row,\footnote{It's a single row in the usual grading, and a single diagonal line with slope
$-1$ in the Adams grading.\index{Adams grading}} all differentials vanish, and this is also the $E_\infty$ page. So
we just have to compute $H^p(G;k)$ for $k\ge 0$.

For example, if $G = \Z/2$ and $k = \F_2$, then $H^*(\Z/2; \F_2) = H^*(\RP^\infty;\F_2) = \F_2[x]$, $\abs x = 1$.
There are no extension issues, since there's only one nonzero term in each total degree. Thus,
\[\pi_{-p}(Hk^{hG}) = H^p(G; k).\]
If you let $G = \Z/2$ and $k$ be any field of odd characteristic, then $H^*(\Z/2; k) = k$ in degree $0$, so the
homotopy groups of $Hk^{h\Z/2}$ are all trivial except for $\pi_0$, which is $k$.
\end{exm}
\begin{rem}
The homotopy fixed points of $X$ are related to the Tate spectrum through a fiber sequence~\eqref{tatedefn}, and
the homotopy fixed point spectral sequence has a similar-looking signature to the Tate spectral sequence
(\cref{tateSS}):\index{Tate spectrum}\index{Tate spectral sequence}
\begin{align*}
E_2^{p,q} &= H^p(G, \pi_q(X))\Longrightarrow \pi_{q-p}(X^{hG})\\
E_2^{p,q} &= \widehat H^p(G; \pi_q(X)) \Longrightarrow \pi_{q-p}(X^{tG}).
\intertext{As Tate cohomology puts group homology and group cohomology together, you might expect there's a third
similar-looking spectral sequence, and you're right:}
E^2_{p,q} &= H_p(G, \pi_q(X))\Longrightarrow \pi_{q-p}(X_{hG})
\end{align*}
This is called the \term{homotopy orbit spectral sequence}.
\end{rem}
\TODO: the cofiber sequence $X_{hG}\to X^{hG}\to X^{tG}$ induces a morphism of multiplicative spectral sequences
from the homotopy fixed point spectral sequence to the Tate spectral sequence. You can use this, as
in~\cite[Prop.~11]{HS14}, to show that $\KR^{tC_2}\simeq *$ and therefore $\KR^{hC_2}\simeq\KR_{hC_2}$. Does the
homotopy orbit spectral sequence fit into this? Would we have to grade it cohomologically?
\begin{exm}
\label{reflection}
Let $C_2$ act on $S^1$ by reflection. Then, $\pi_i(S^1)$ is trivial unless $i = 1$, in which case we get $\Z$.
Hence,
\[E_2^{p,q} = \begin{cases}
	H^p(C_2;\Z), &q = 1\\
	0, &\text{otherwise}.
\end{cases}\]
Under the isomorphism $\Z[C_2]\cong \Z[x]/(x^2-1)$, the $\Z[C_2]$-module structure on $\Z$ is the map
$\Z[C_2]\to\Z$ sending $x\mapsto -1$, i.e.\ $C_2$ acts on $\Z$ through the nontrivial action. We'll let $\Z_\sigma$
denote $\Z$ with this action, and $\Z$ denote the integers with the trivial $C_2$-action. To compute the group
cohomology, we need to compute a free resolution $P_\bullet\to\Z$ as a \emph{trivial}
$\Z[C_2]$-module:\footnote{\TODO: why was this again?}
\[\xymatrix{
	\dotsb\ar[r] & \Z[C_2]\ar[r]^{\cdot(x-1)}\ar[r] & \Z[C_2]\ar[r]^{\cdot(x+1)} & \Z[C_2]\ar[r]^{\cdot(x-1)} &
	\Z[C_2]\ar[r]^-{x\mapsto 1} & \Z\ar[r] & 0.
}\]
Now we compute $\Hom_{\Z[C_2]}(P_\bullet, \Z_\sigma)$:
\[\xymatrix{
	\dotsb & \Z\ar[l] & \Z\ar[l]_{-2} & \Z\ar[l]_0 & \Z.\ar[l]_{-2}
}\]
Taking homology, we conclude that
\[H^p(C_2, \Z_\sigma) = \begin{cases}
	\Z/2, &p > 0\text{ odd}\\
	0, &\text{otherwise.}
\end{cases}\]
The spectral sequence degenerates at page $2$, but we haven't yet calculated $\pi_*((S^1)^{hC_2})$ ---
$(S^1)^{hC_2}$ is a space, so cannot have negative-degree homotopy groups. But since this is a fringed spectral
sequence, the stuff in negative degrees doesn't apply to the calculation of homotopy groups, and $q - p = 0,1$
mixes together in a complicated way. In this case, it tells us that $\pi_0((S^1)^{hC_2}) = \Z/2$ and the higher
homotopy groups vanish.\footnote{\TODO: can we calculate $(S^1)^{hC_2}$ explicitly and see this?}
\end{exm}
\begin{exm}
Atiyah~\cite{AtiyahKR} showed that complex conjugation places a $C_2$-equivariant structure on $\KU$; the resulting
$C_2$-spectrum is called \term*{$\KR$-theory}, and denoted $\KR$. Its homotopy fixed points are real
(nonequivariant) $K$-theory:\index{K-theory@$K$-theory!real equivariant
$K$-theory}\index{KR-theory@$\KR$-theory|see {real $K$-theory}}
\begin{thm}
$\KR^{hC_2} = \KO$.
\end{thm}
Playing with the homotopy fixed-point spectral sequence for $\KR$, as in~\cite{HS14}, is a good way to become more
familiar with it. \TODO: let's say more about this.
\end{exm}
\begin{ques}
It should be possible to do something similar with $\mathit{KSp}$ and $i\mapsto -i$, $j\mapsto -j$, and $k\mapsto
-k$. If we take homotopy fixed points with respect to the last two, do we end up with $\KU$? If we take homotopy
fixed points with respect to all three, do we get $\KO$?
\end{ques}
\begin{ex}
The analogous result doesn't hold for Eilenberg-Mac Lane spectra. Let $C_2$ act on $\Z[i]$ by complex conjugation,
making $\Z[i]$ into a $\Z[C_2]$-module. This data determines a Mackey functor $\underline{\Z[i]}$ and therefore an
Eilenberg-Mac Lane $C_2$-spectrum $H\underline{\Z[i]}$ as in \cref{EMmodel}.\index{equivariant Eilenberg-Mac Lane
spectrum}

Though $\Z[i]^{C_2} = \Z$, run the homotopy fixed-point spectral sequence to show that
$H\underline{\Z[i]}^{hC_2}\not\simeq H\Z$.
\end{ex}
There are some other examples of calculations with this spectral sequence, including Bruner-Greenlees~\cite{BG10},
Hill-Hopkins-Ravenel~\cite{HHREO2, HHR}, and Hahn-Shi~\cite{HS17}. Greenlees~\cite{GreenCalc}, working over $C_2$,
introduces an $\RO(C_2)$-graded version of the homotopy fixed point spectral sequence and applies it to
$H\underline\Z$ and connective real $K$-theory $\mathit{kr}$.\index{connective real $K$-theory}
