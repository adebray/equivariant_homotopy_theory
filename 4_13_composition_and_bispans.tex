\label{tambara_functor}
Green functors are pretty cool, but we can and should expect more. Here's one reason. Suppose $R$ is an ordinary
ring, or even a semiring, and let's look at the structure on $\set{R^X}$ (here, $R^X\coloneqq \Map(X,R)$) for some
sets $X$. A map $X\to Y$ produces two kinds of maps $t, n\colon R^X\to R^Y$. The first, a transfer-like map, is
defined by the formula
\[t(f)(y)\coloneqq \sum_{f(x) = y} r(x),\]
but we could also do
\[n(f)(y)\coloneqq \prod_{f(x) = y} r(x),\]
and there's no analogue of this in Green functors. The replacements are called \term[Tambara functor]{Tambara
functors} (or \term*{TNR functors}).\footnote{TNR stands for ``transfer, norm, restriction.''}\index{TNR
functor|see {Tambara functor}} Strickland's paper~\cite{StricklandTambara} on Tambara functors is really nice: it's
complete and careful. Lewis has unpublished notes~\cite{LewisGreen} on Green functors, which are also good; you can
find them, along with almost everything else in equivariant homotopy theory, on Doug Ravenel's website. Lewis'
notes owe a debt to McClure's unpublished notes, which have been lost to history.
\begin{rem}
This is related to the theory of \term[polynomial functor]{polynomial functors}, which we won't talk about. You can
set this up in great abstraction; if you torture it, you can get it to output actual polynomials. Tambara functors
are an example of polynomial functors.
\end{rem}
\begin{defn}
Let $\fC$ be a locally Cartesian closed category and $X,Y\in\fC$. Then, the \term[bispan]{category of bispans} from
$X$ to $Y$, denoted $\Bispan_\fC(X,Y)$ is the category whose objects are diagrams
\begin{equation}
\label{span1}
\xymatrix{
	X & S\ar[l]\ar[r] & T\ar[r] & Y
}
\end{equation}
and whose morphisms are commutative diagrams
\[\xymatrix@R=0.2cm{
	& S\ar[r]\ar[dl]\ar[dd]^\cong & T\ar[dd]^\cong\ar[dr]\\
	X &&& Y\\
	& S'\ar[r]\ar[ul] & T'.\ar[ur]
}\]
\end{defn}
\begin{exm}
\label{suggest_name}
We'll care in particular about a few suggestively named bispans associated to a $\fC$-morphism $f\colon S\to T$:
\begin{enumerate}
	\item Let $R_f$ denote the bispan
	\[\xymatrix{
		T & S\ar[l]_f\ar[r]^\id & S\ar[r]^\id & S.
	}\]
	\item Let $N_f$ denote the bispan
	\[\xymatrix{
		S & S\ar[l]_\id\ar[r]^f & T\ar[r]^\id & T.
	}\]
	\item Let $T_f$ denote the bispan
	\[\xymatrix{
		S & S\ar[l]_\id\ar[r]^\id & S\ar[r]^f & T.\qedhere
	}\]
\end{enumerate}
\end{exm}
$R_f$ and $T_f$ will encode the restriction and transfer maps, respectively, in the underlying Mackey functor of a
Tambara functor.\index{restriction map}\index{transfer map!for a Tambara functor}

One interesting question about bispans is when you can compose in the objects: is there a ``composition map''
\begin{equation}
\label{compbispan}
\Bispan_\fC(Y,Z)\times\Bispan_\fC(X,Y)\longrightarrow\Bispan_\fC(X,Z)?
\end{equation}
To get this, we'll need to actually look at what being a locally Cartesian closed category means.\index{locally
Cartesian closed category} Specifically, it means that any map $f\colon X\to Y$ in $\fC$ induces a pullback
$f^*:\fC_{/Y}\to\fC_{/X}$, and we require that this map has a left and a right adjoint, respectively called the
\term{dependent sum} $\Sigma_f$ and the \term{dependent product} $\Pi_f$, respectively. These can be thought of
(e.g.\ in $\Set$) as coproducts (resp.\ products) indexed by the fibers.
\begin{rem}
There's an interesting connection to homotopy type theory: one of the early big theorems in type theory is that
propositional calculus and lambda calculus are formally equivalent to the theory of locally Cartesian closed
categories, so there was a lot of research into importing these categorical notions into type
theory.\index{homotopy type theory}\index{lambda calculus}\index{propositional calculus}

The dependent sum and product are examples of base-change functors $f_!$ and $f_*$; the given notation may be
unfamiliar, but is standard in the category-theoretic literature.
\end{rem}
\begin{exm}
Let's see what this means in $G\Set$. For a map $f\colon X\to Y$, we want to compute $\Pi_f$. If $q\colon A\to Y$
is an object of $G\Set_{/X}$, then its dependent product is $q'\colon \Pi_fA\to Y$, where
\[\Pi_f A = \set{(y,s)\mid y\in Y, s\colon f^{-1}(y)\to A\text{ such that } q\circ s = \id}.\]
That is, it's elements of $y$ along with sections of $q$ above the fiber $f^{-1}(y)$.
\end{exm}
Tambara functors will be associated to a particular kind of diagram.
\begin{defn}
An \term{exponential diagram} in a locally Cartesian closed category $\fC$ is any diagram isomorphic\footnote{This
means in the category of diagrams of this shape, i.e.\ there are isomorphisms for every object in the diagram that
commute with the maps in the diagram.} to one of the form
\begin{equation}
\label{exponential}
\gathxy{
	X\ar[d]^f & A\ar[l]_-q & X\times_Y \Pi_f A\ar[l]_-{\mathit{ev}}\ar[d]^{\pi_2}\\
	Y & \Pi_f A\ar[l]_-{q'} & \Pi_fA.\ar@{=}[l]
}
\end{equation}
This commutes.
\end{defn}
We can use this to build a diagram associated to two spans $X\gets A\to B\to Y$ and $Y\gets C\to D\to Z$: start
with
\begin{equation}
\label{snake}
\gathxy{
	X & A\ar[l]\ar[d]\\
	& B\ar[d]\\
	& Y & C\ar[l]\ar[d]\\
	& Z & D.\ar[l]
}
\end{equation}
\TODO: in class, we got confused and didn't finish writing this diagram down. Anyways, you can use this (and in
particular the exponential map) to define the composition operation we asked for in~\eqref{compbispan}.
% bigdiag
% (1), (2), (3) pullbacks
% (4) is an exponential diagram
If we let $0$ denote the bispan
\[\xymatrix{
	X & \emptyset\ar[l]\ar[r] & \emptyset\ar[r] & Y
}\]
and $1$ denote the bispan
\[\xymatrix{
	X & \emptyset\ar[l]\ar[r] & Y\ar[r]^\id & Y
}\]
we get a semiring structure on the set of isomorphism classes of bispans. The sum of $[X\gets S\to T\to Y]$ and
$[X\gets S'\to T'\to Y]$ is
\[\xymatrix{
	X & S\amalg S'\ar[l]\ar[r] & T\amalg T'\ar[r] & Y,
}\]
and their product is
\[\xymatrix{
	X & (S\times_Y T') \amalg (S'\times_Y T)\ar[l]\ar[r] & T\times_Y T'\ar[r] & Y,
}\]
and you can check $0$ and $1$ are the identities for the sum and product, respectively. When you see a semiring,
you might think to take its Grothendieck group, and we will do this. We'll eventually also consider ``bispans with
coefficients,'' i.e.\ those bispans $X\gets S\stackrel f\to T\to Y$ where $f$ is constrained to a subcategory $\fD$
of $\fC$. You need a condition on $\fD$ for this to be a category,\footnote{The condition is that $\fD$ is
\index{wide subcategory}\term*{wide} (i.e.\ contains all objects in $\fC$), closed under pullbacks, and closed
under coproducts. \TODO: I might have gotten an axiom wrong.} which is exactly the same as the condition for an
indexing system! So there's $N_\infty$-operads floating around, and hence from a categorical perspective, we're
forced to consider this notion of multiplication.\index{Ninfinity-operad@$N_\infty$-operad}

%We'll then discuss Tambara functors and how the multiplicative structure on $\pi_0$ of a ring spectrum is a Tambara
%functor. Then, we'll move to discussing the norm functor and some of its myriad uses.
\begin{rem}
Spans (sometimes also called correspondences) are useful in many different contexts. That bispans come up is
weirder: they're certainly less ubiquitous. On the other hand, when you want one wrong-way map and two right-way
maps, it does seem like a good idea.
\end{rem}
One can also use the category of bispans can be used to understand polynomial functors, but the papers on
polynomial functors are written in the language of type theory and things internal to the relevant topos. This is
fine, but it may be unfamiliar.

Given the bispan~\eqref{span1} in the category of sets, we obtain a sequence of functors
\[\xymatrix{
	\Set_{/X}\ar[r]^{f^*} & \Set_{/S}\ar[r]^{\Pi_g} & \Set_{/T}\ar[r]^{\Sigma_h} & \Set_{/Y}.
}\]
Their composition is called a \term{polynomial functor}: it's a sum of a product of ``monomials.'' The sum and
product are the dependent ones we defined above and can be a little confusing, but $f^*$ is pullback and $\Sigma_h$
is composition, so that's not so bad.

We'd like to compose bispans. Last time, we defined exponential diagrams to be those isomorphic to diagrams of the
form~\eqref{exponential}.
\begin{ex}
Show that an exponential diagram is a pullback diagram.
\end{ex}
We can use this to define the composition of two bispans $[Y\gets C\stackrel f\to D\to Z]\circ[X\gets A\to B\to
Y]$. We start by superimposing the two diagrams as in~\eqref{snake}, then letting $B'\coloneqq B\times_Y C$ and
$A'\coloneqq A\times_B B'$:
\[\xymatrix{
	X & A\ar[l]\ar[d] & \textcolor{blue}{A'}\ar@[blue][l]\ar@[blue][d]\\
	& B\ar[d] & \textcolor{blue}{B'}\ar@[blue][l]\ar@[blue][d]\\
	& Y & C\ar[l]\ar[d]^f\\
	& Z & D.\ar[l]
}\]
Now we can form the exponential diagram for $B'\to C\to D$:
\[\xymatrix{
	X & A\ar[l]\ar[d] & A'\ar[l]\ar[d]\\
	& B\ar[d] & B'\ar[l]\ar[d] & \textcolor{blue}{C\times_D\Pi_fB'}\ar@[blue][l]\ar@[blue][dd]\\
	& Y & C\ar[l]\ar[d]^f\\
	& Z & D\ar[l] & \textcolor{blue}{\Pi_fB'}.\ar@[blue][l]
}\]
Finally, let $\widetilde A\coloneqq A'\times_{B'} (C\times_D\Pi_fB')$:
\[\xymatrix{
	X & A\ar[l]\ar[d] & A'\ar[l]\ar[d] & \textcolor{blue}{\widetilde A}\ar@[blue][l]\ar@[blue][d]\\
	& B\ar[d] & B'\ar[l]\ar[d] & C\times_D\Pi_fB'\ar[l]\ar[dd]\\
	& Y & C\ar[l]\ar[d]^f\\
	& Z & D\ar[l] & \Pi_fB'.\ar[l]
}\]
Now, we can define the \term[composition of bispans]{composition} of these two bispans to be the bispan $[X\gets
\widetilde A\to\Pi_fB'\to Z]$.

We'd like this to actually behave like a composition.
\begin{lem}
Composition with the \term{identity bispan}
\[\xymatrix{
	Y & Y\ar[l]_\id\ar[r]^\id & Y\ar[r]^\id & Y
}\]
is the identity, up to natural isomorphism.
\end{lem}
\begin{proof}[Partial proof]
Let $[X\stackrel f\gets A\stackrel g\to B\stackrel h\to Y]$ be some other bispan. Then, $\Pi_\id B\cong B$, and
pulling back by the identity acts as the identity, so $B' = B$ and $A' = A$. Thus, $Y\times_Y\Pi_\id B \cong
\Pi_\id B = B$ and $\widetilde A = A$ (again, we're pulling back by the identity map), so we obtain the diagram
\[\xymatrix{
	X & A\ar[l]_f\ar[d]^g & A\ar[l]_\id\ar[d]^g & A\ar[l]_\id\ar[d]^g\\
	& B\ar[d]^h & B\ar[l]_\id\ar[d]^h & B\ar[l]_\id\ar[dd]\\
	& Y & Y\ar[l]_\id\ar[d]^\id\\
	& Y & Y\ar[l]_\id & B.\ar[l]_h
}\]
and therefore the composition is $[X\gets A\to B\to Y]$ again, at least up to natural isomorphism.
\end{proof}
\begin{ex}
Finish the proof by checking composition with the identity on the right.
\end{ex}
\begin{rem}
We're working towards showing that isomorphism classes of bispans are the morphisms in a category. If you don't
like taking isomorphism classes, you can use bispans on the nose, in which case you'll get a bicategory. If you
like this, it's an exercise to write this out carefully. In general, since working through this argument (at either
categorical level) is a useful exercise, it's not written down explicitly: if you work through this stuff, feel
free to add it to the notes.

That said, associativity is going to be a chore no matter how you go about it.
\end{rem}
Last time in \cref{suggest_name}, we defined three bispans $R_f$, $N_f$, and $T_f$ associated with a map $f\colon
S\to T$.
\begin{lem}
\label{compasNTR}
$[X\stackrel f\gets A\stackrel g\to B\stackrel h\to Z]$ is isomorphic to the composition $T_h\circ N_g\circ R_f$.
\end{lem}
\begin{proof}[Partial proof]
Recall $R_f = [X\stackrel f\gets S\stackrel\id\to S\stackrel\id\to S]$ and $N_g = [S\stackrel\id\gets S\stackrel
g\to T\stackrel\id\to T]$. We'll show their composition is
\begin{equation}
\label{firstcomp}
\xymatrix{
	X & S\ar[l]_f\ar[r]^g & T\ar[r]^\id &T.
}
\end{equation}
Pullbacks by the identity are the identity, so we can start with
\[\xymatrix{
X & S\ar[l]_f\ar[d]^\id & S\ar[l]_\id\ar[d]^\id\\
& S\ar[d]^\id & S\ar[l]_\id\ar[d]^\id\\
& S & S\ar[l]_\id\ar[d]^g\\
& T & T.\ar[l]_\id
}\]
We claim $\Pi_gS\cong T$, which we'll check momentarily; under this assumption, $S\times_T \Pi_g S\cong S$, so the
full diagram is
\[\xymatrix{
X & S\ar[l]_f\ar[d]^\id & S\ar[l]_\id\ar[d]^\id &S\ar[l]_\id\ar[d]_\id\\
& S\ar[d]^\id & S\ar[l]_\id\ar[d]^\id & S\times_T\Pi_gS\cong S\ar[dd]^g\ar[l]_-\id\\
& S & S\ar[l]_\id\ar[d]^g\\
& T & T\ar[l]_\id & \Pi_g S\cong T,\ar[l]_\id
}\]
and the composition is~\eqref{firstcomp}, as desired.

Now let's prove the claim. Given an $f\colon X\to Y$ and an $A\to X$, $\Pi_f A$ fits into a pullback
\begin{equation}
\label{Pif}
\gathxy{
	\Pi_f A\ar[r]\ar[d] & \Map_T(X,X)\ar[d]\\
	Y\ar[r] & \Map_T(A,X).
}
\end{equation}
%This is a kind of fiberwise mapping space: $\Map_T(X,X)_b\cong \Map(X_b, X_b)$, But you have to topologize this,
%and this is one place where point-set technicalities cannot be avoided, and you have to worry about whether your
%spaces are compactly generated and weakly Hausdorff. Suffice to say this has been worked out; we won't worry about
%those details.

From~\eqref{Pif}, $\Pi_g S$ is the space of functions $s\colon g^{-1}(t)\to S$ such that $\id\circ s(x) = x$, and
this space can be identified with $T$.

The second step of the proof, composing with $T_h$, is analogous.
\end{proof}
\begin{prop}\hfill
\label{NTRcomp}
\begin{enumerate}
	\item $N_g\circ N_{g'} = N_{gg'}$.
	\item\label{hh'} $T_h\circ T_{h'} = T_{hh'}$.
	\item $R_f\circ R_{f'} = R_{f'f}$.
\end{enumerate}
\end{prop}
\begin{proof}[Proof of part~\eqref{hh'}]
Explicitly, we want to compose $X\stackrel\id\gets X\stackrel{g'}\to Y\stackrel\id\to Y$ and $X\stackrel\id\gets
X\stackrel{g}\to Y\stackrel\id\to Y$. We can fill in the pullbacks on the left immediately:
\[\xymatrix{
	X & X\ar[l]_\id\ar[d]^{g'} & X\ar[l]_\id\ar[d]^{g'}\\
	& Y\ar[d]^\id & Y\ar[l]_\id\ar[d]^\id\\
	& Y & Y\ar[l]_\id\ar[d]^g\\
	& Z & Z.\ar[l]_\id
}\]
To form the exponential diagram, observe that by the same argument as in the proof of \cref{compasNTR}, $\Pi_g
Y\cong Z$, and therefore $Y\times_Z\Pi_g Y\cong Y$. Thus, the finished diagram is
\[\xymatrix{
	X & X\ar[l]_\id\ar[d]^{g'} & X\ar[l]_\id\ar[d]^{g'} & X\ar[l]_\id\ar[d]^{g'}\\
	& Y\ar[d]^\id & Y\ar[l]_\id\ar[d]^\id & Y\ar[l]_\id\ar[dd]^g\\
	& Y & Y\ar[l]_\id\ar[d]^g\\
	& Z & Z\ar[l]_\id & Z\ar[l]_\id,
}\]
so the composition really is $T_{hh'}$.
\end{proof}
Again, it's instructive to fill in the other two parts, which have similar proofs.

The next proposition is harder.
\begin{prop}
Given a pullback diagram
\[\xymatrix{
	X'\ar[r]^{g'}\ar[d]^{f'} & X\ar[d]^f\\
	Y'\ar[r]^g & Y,
}\]
then
\begin{enumerate}
	\item $R_f\circ N_g = N_{g'}\circ R_{f'}$, and
	\item $R_f\circ T_g = T_{g'}\circ R_{f'}$.
\end{enumerate}
\end{prop}
\begin{prop}
Given an exponential diagram
\[\xymatrix{
	X\ar[d]^g & A\ar[l]_h & X\times_Y \Pi_g A\ar[l]_-{f'}\ar[d]^{g'}\\
	Y && \Pi_g A,\ar[ll]_-{h'}
}\]
$N_g\circ T_h = T_{h'}\circ N_{g'}\circ R_{f'}$.
\end{prop}
With all of these in place, we get what we're looking for.
\begin{cor}
The category $P(\fC)$ whose objects are objects of $\fC$ and morphisms are isomorphism classes of bispans really is
a category, in that composition of bispans satisfies the necessary axioms.
\end{cor}
\TODO: I missed the thing right after this.

For a reference on this stuff, check out Tambara's original paper~\cite{Tambara}. This is really recent, from the
early 1990s, and this business about using bispans to encode right-way and wrong-way maps is not as well explored
as they could be.
\begin{rem}
This category of bispans is reminiscent of the Dwyer-Kan hammock localization~\cite{Hammock}, and it may be
possible to obtain bispans from the hammock localization of a particular category with weak equivalences. Is this
useful? Maybe.\index{Dwyer-Kan localization}
\end{rem}
If $\fC = G\Set$, then we'll let $\tilde{P}^G\coloneqq P(\fC)$. The hom sets of $\tilde{P}^G$ are commutative
monoids under direct sum. Taking the Grothendieck group of each hom-set (as we did with Mackey functors), we obtain
a category $P^G$, sometimes called the \term{bispan category}.
\begin{defn}
A \term{Tambara functor} is an additive functor $P^G\to\Ab$.
\end{defn}
\begin{comp}{rem}{enumerate}
	\item Just as you can think of $G$-spectra as spectrally enriched Mackey functors, you can think of $G$-ring
	spectra as spectrally enriched Tambara functors. This is harder to make precise, but is still a useful guiding
	analogy. For connective spectra, this is discussed in~\cite{Hoyer}.\index{presheaf!on the bispan category}
	\item The algebra of Tambara functors is still under construction. Nakaoka has a few recent
	papers~\cite{NakaokaIdeals, NakaokaFractions, NakaokaSpectrum} about, e.g. ideals and modules for Tambara
	functors. This is closely related to the project of equivariant derived algebraic geometry, which would relate
	to the algebraic geometry of Tambara functors. The questions are interesting and complicated --- how do you
	localize? What happens when you do?  There are probably algebraic questions in this area that aren't too
	difficult to pose and answer, simply because there aren't that many people looking.
	\item Just as we described Mackey functors as a pair of a covariant and a contravariant functor satisfying a
	push-pull axiom, there's a similar (more complicated) definition for Tambara functors, with three functors
	satisfying some interoperability conditions. We won't need this for the time being, so we're not going to write
	it out.
	\item If you forget the norm map, a Tambara functor defines a Green functor. A Green functor may extend to a
	Tambara functor, but such an extension need not exist or be unique~\cite{MazurThesis}.\index{Green functor}
	\qedhere
\end{comp}
\begin{exm}
Let $R$ be a ring with a $G$-action. Then, $\Map_G(\bl,R)$ is a Tambara functor. In this case, given a map $f\colon
X\to Y$, the descriptions of $f^*$, $\Sigma_f$, and $\Pi_f$ are pretty concrete:
\begin{align*}
	f^*(\theta)(x) &= \theta(f(x))\\
	\Sigma_f(\theta)(x) &= \sum_{x\in f^{-1}(y)} \theta(x)\\
	\Pi_f(\theta)(x) &= \prod_{x\in f^{-1}(y)} \theta(x).\qedhere
\end{align*}
\end{exm}
\begin{exm}[Burnside Tambara functor]
The Burnside ring continues to keep giving. Let $B_G(X)$ denote the category of $G$-sets over $X$. Given a map
$f\colon X\to Y$ of sets, $f^*(B\to Y)$ is $X\times_Y B$, $\Sigma_f(A\to X)$ is $A\to X\to Y$, and $\Pi_f(A\to X) =
\Pi_fA\to X$.\index{Burnside Mackey functor!as a Tambara functor}
\end{exm}
\begin{exm}\hfill
\label{Evensexm}
\begin{enumerate}
	\item Group cohomology is a Tambara functor: the transfer and restriction maps are what we've seen before, and
	the norm is the \term{Evens multiplicative transfer}, which we'll talk about in \S\ref{evenssec}.\index{group
	cohomology!as a Tambara functor}
	\item Let $X$ be a commutative ring object in $\Sp^G$. Then, $\pi_0(X)$ is a Tambara functor. The idea is that
	it'll be an algebra over an $N_\infty$ operad over a ``complete'' indexing system (again, this will be
	elaborated on).
	\item Let $\fD\subseteq G\Set$ be a wide subcategory, meaning it has all the objects (but perhaps not all the
	morphisms). Then, let $P_\fD^G$ denote the category of bispans $X\stackrel f\gets A\stackrel g\to B\stackrel
	h\to Y$ where $g$ is in $\fD$. We'd like $P_\fD^G$ to be a category, and there's a criterion that's not too
	hard to check.\qedhere
\end{enumerate}
\end{exm}
\begin{prop}
$P_\fD^G$ is a category if $\fD$ is \term*{stable under pullback}, i.e.\ for any pullback diagram\index{stable
under pullback|see {pullback stable}}
\[\xymatrix{
	Q\ar[r]\ar[d]^f & Z\ar[d]^g\\
	X\ar[r] & Y,
}\]
if $g\in\fD$, then $f\in\fD$.
\end{prop}
One also says that $\fD$ is \term{pullback stable} or \term*{closed under base change}.
\begin{proof}[Proof sketch]
The argument amounts to showing that $(T_h N_g R_f)(T_{h'}N_{g'}R_{f'})$ can be put in some normal form where the
$N$ piece is obtained from something in $\fD$. We mostly only have to worry about interchanging $N$ with things:
since $\fD$ is a category, $N_gN_{g'} = N_{gg'}$ keeps us inside $P_\fD^G$.

We use the pullback stability to address the exponential diagram
\[\xymatrix{
	X\ar[d]^f & A\ar[l] & X\times_Y\Pi_fA\ar[d]^\vp\ar[l]\\
	Y && \Pi_f A.\ar[ll]
}\]
This is a pullback diagram, so if $f\in\fD$, then so is $\vp$.
\end{proof}
Since $G$-sets are sums of orbits, we want to be able to work with sums of orbits as we've done before. Let
$\Orb_\fD$ be the full subcategory of $\fD$ on the orbits $\set{G/H}$.
\begin{defn}
We say that $\fD$ is \term{coproduct complete} if coproducts in $\fD$ are the same as those in $\fC$, and $\fD$ is
closed under coproducts.
\end{defn}
\begin{lem}
If $\fD$ is coproduct complete, then $\fD$ is the coproduct completion of $\Orb_\fD$.
\end{lem}
That is, we start with orbits, throw in coproduct, and obtain everything, which is nice.
\begin{defn}
Let $\fD$ be a wide, pullback stable, coproduct complete subcategory of $\fC$. Then, a \term*{$\fD$-Tambara
functor} is a functor $P_\fD^G\to\Ab$. In~\cite{BlumbergHill}, these are called
\term[incomplete Tambara functor]{incomplete Tambara functors}.
\index{D-Tambara functor@$\fD$-Tambara functor|see {incomplete Tambara
functor}}
\end{defn}
We won't be able to delve into the following proof.
\begin{thm}[\cite{BlumbergHill}]
There is an equivalence of categories between the poset of wide, pullback stable, coproduct complete subcategories
of $G\Set$ and the category of indexing systems for $G$.
\end{thm}
This tells us that all of the conditions imposed on indexing systems are inevitably imposed on us if we want
equivariant ring spectra.

Looking forward, we'll discuss the Evens norm and $\pi_0$ of ring spectra as promised in \cref{Evensexm}, then
proceed to the norm as defined in~\cite{HHR}: what its structure is, what you do with it, and why it was useful in
their argument. Once we've finished with that, we've almost arrived at the frontier of equivariant stable homotopy
theory.

To the degree that you can fill in the details, this is a large subset of what people know about equivariant stable
homotopy theory. Fifteen years ago, very few computations were known, and the computations were hard. But the
development of the slice filtration and slice spectral sequence is a recent innovation, used to attack the Kervaire
invariant one problem, and has led to lots of computations. We won't discuss this in detail, but it will probably
lead to further useful and interesting computations.
