\begin{quote}\textit{
	``One shift ($\Sigma$), two shift ($\Sigma^2$), red shift ($K$), blue shift ($(\bl)^{tG}$).''\index{pun}
}\end{quote}
Tate spectra are the spectral analogue of Tate cohomology in group cohomology. This is yet another instance of a
fruitful phenomenon: \emph{group cohomology is an important source of inspiration in equivariant homotopy theory}.
By looking at group cohomology in the right way, we found Mackey functors, the Wirthmüller isomorphism, norms,
Tambara functors, and more (though some of these were hard to see from group cohomology first). Group cohomology is
a nice testing ground, because of its concreteness: there are elements and you can add and multiply
things!\index{group cohomology}
\begin{rem}
The definitive reference on Tate spectra is Greenlees-May~\cite{GeneralizedTate}, which builds on~\cite{ACD}. The
latter is lower-tech, but is more direct and has nice ideas. Another good reference is~\cite[\S 4]{HM03}, a section
of Hesselholt-Madsen's paper about the algebraic $K$-theory of local fields.\index{algebraic $K$-theory}

The reason you can find Tate spectra in an \latin{a priori} unrelated paper is that they are important for
computing algebraic $K$-theory via trace methods: it's possible to compute algebraic $K$-theory from topological
cyclic homology and topological Hochschild homology, and the computation passes through iterated Tate spectral
sequences. The best examples for the use of Tate spectra were until quite recently from the literature on trace
methods, e.g.\ almost all of Hesselholt-Madsen's papers.\index{trace methods}\index{topological cyclic
homology}\index{topological Hochschild homology}

There's also a connection between Tate spectra and chromatic homotopy theory: like the redshift conjecture asserts
that algebraic $K$-theory raises chromatic height, Tate spectra seem to have blueshift, lowering by a chromatic
level.\index{chromatic homotopy theory}\index{redshift conjecture}\index{blueshift conjecture}
\end{rem}
Let $M$ be a $k[G]$-module. Then, its \term{algebraic orbits} are $M_G\coloneqq k\otimes_{k[G]} M$, and its
\term{algebraic fixed points} are $M^G\coloneqq\Hom_{k[G]}(k, M)$. We'll define a map $N\colon M_G\to M^G$, which
is called a norm map, but it's more like a transfer map. Namely, there's a distinguished element in $k[G]$,
$N_G\coloneqq \sum_{g\in G} g$, and $N$ is multiplication by $N_G$:\index{norm map!for $k[G]$-modules}
\[N\colon \overline x\mapsto\sum_{g\in G} gx.\]
This lands in $M$, but is clearly a fixed point.
\begin{ex}
Show that this is independent of the choice of representative for $\overline x\in M_G$.
\end{ex}
This is slightly less obvious, but not hard.

We'll think of $N$ as a map $H_0(G;M)\to H^0(G;M)$, which ``sews together'' group homology and group cohomology
with coefficients in $M$. Nonetheless, it's an interesting object in its own right. What we'll get is
\begin{subequations}
\begin{equation}
\label{mosttate}
\begin{gathered}
\tH^i(G; M) = \begin{cases}
	H^i(G;M), &i\ge 1\\
	H_{-i-1}(G;M), &i\le -2,
\end{cases}
\end{gathered}
\end{equation}
and for $i = 0,1$ there's an exact sequence
\begin{equation}
\label{resttate}
\xymatrix{
	0\ar[r] & \tH^{-1}(G;M)\ar[r] & H_0(G;M)\ar[r]^N & H^0(G;M)\ar[r] &\tH^0(G;M)\ar[r] & 0.
}
\end{equation}
\end{subequations}
You can build this by forming \term[complete resolution]{complete resolutions}: beginning with a pair $P_*\to k$
and $k\to I_*$, respectively a projective and an injective resolution of $k$ as a $k[G]$-module, and you can sew
these together by truncating $I_*$ and dualizing to $k\to\overline P_*$, then connecting them along
$k$.\footnote{\TODO: make
precise.}

Another construction is to let $\widetilde P_*$ be the mapping cone of a projective resolution $P_*\to k$ of $k$ as
a $k[G]$-module. This extends to a sequence
\[\xymatrix{
	P_*\ar[r] & k\ar[r] & \widetilde P_*\ar[r] & \Sigma P_*\ar[r] & \dotsb
}\]
\begin{defn}
Using this, define the \term{Tate cohomology} $\tH^*(G;M)$ to be the cohomology of the complex
\[(\widetilde P_*\otimes\Hom(P_*, M))^G.\]
\end{defn}
To see that this agrees with~\eqref{mosttate} and~\eqref{resttate}, use the long exact sequence and a comparison
\[(P_*\otimes\Hom(P_*, M))^G\cong (P_*\otimes M)_G\cong (P_*\otimes M)^G.\]
We'd like to imitate this construction topologically. The analogue of the projective resolution of $k$ is $EG_+\to
S^0$ which collapses everything to the non-basepoint. Let $\widetilde{EG}$ be the cofiber of this map. Thus we get
a map
\[X\cong F(S^0, X)\longrightarrow F(EG_+, X),\]
and thus obtain a diagram of cofiber sequences:
\begin{equation}
\label{pretate}
\gathxy{
	EG_+\wedge X\ar[r]\ar[d] & X\ar[d]\ar[r] & \tEG\wedge X\ar[d]\\
	EG_+\wedge F(EG_+, X)\ar[r] & F(EG_+, X)\ar[r] & \tEG\wedge F(EG_+, X).
}
\end{equation}
Using the Adams isomorphism,\footnote{We haven't discussed the Adams isomorphism, but you can also show this
directly: the idea is that smashing with $EG_+$ makes things free, hence you just have to check on underlying
spaces, and taking $F(EG_+,\bl)$ makes it cofree, so you can work with that.} you can check that\index{Adams
isomorphism}
\[(EG_+\wedge X)^G\cong X_{hG}.\]
Now, apply $(\bl)^G$ to~\eqref{pretate}, and you obtain
\begin{equation}
\label{tatedefn}
\gathxy{
	X_{hG}\ar[r]\ar[d]^\cong & X^G\pullback\ar[d]\ar[r] & \Phi^GX\ar[d]\\
	X_{hG}\ar[r]^N & X^{hG}\ar[r] & X^{tG}.
}
\end{equation}
The map $N$ is the canonical orbits-to-fixed-points map. The spectrum in the lower right, $X^{tG}$, is called the
\term{Tate spectrum} of $X$.\index{orbits-to-fixed-points map}
\begin{comp}{rem}{enumerate}
	\item The fact that the right-hand square is a pullback diagram means that you can recover $X^G$ from $\Phi^G
	X$. For example, if $G = C_p$, so there's only one nontrivial subgroup, you can describe the homotopy theory of
	$G$-spectra diagramatically, since geometric fixed points capture weak equivalences. This philosophy is put to
	work in Saul Glassman's papers.
	\item The right-hand square looks like an arithmetic square or fracture square, and this is a perspective
	worth taking seriously. \qedhere
\end{comp}
\begin{defn}
The \term*{free homotopy type} or \term{Borel homotopy type} of a $G$-spectrum $X$ is the homotopy type of
$EG_+\wedge X$.\footnote{Borel homotopy types capture the naïve notion of equivariant spectra, namely spectra with
a $G$-action or spectra over $BG$. They have some nice properties, e.g.\ their homotopy fixed points are the same
as their ordinary fixed points~\cite[Prop.~6.19]{MNN17}, and are often useful outside homotopy theory, such as
in~\cite{FreedHopkins}.}\index{free homotopy type|see {Borel homotopy type}}\index{homotopy fixed points!of a naïve
$G$-spectrum}
\end{defn}
If you make the action free, you lose some information, but often enough remains to be useful. In particular, the
bottom row of~\eqref{tatedefn} is a cofiber sequence, so $X^{tG}$ only depends on the free homotopy type of $X$.
\begin{rem}
Since $EG_+$ is a space, it has a diagonal, and $(EG\times EG)_+\cong EG_+$, so it's a ring, and therefore
$(\bl)^{tG}$ is lax monoidal. It's a bit of a chore to make this precise.

The Tate spectrum construction is not functorial in $G$; however, it does define a Mackey
functor~\cite{GeneralizedTate}.\index{Mackey functor}
\end{rem}
This construction now leads to the Tate spectral sequence.\footnote{In this argument, we'll make nice (co)fibrancy
assumptions, e.g.\ that $X$ is fibrant and $EG_+$ is cofibrant; a careful treatment would make the (co)fibrant
replacements explicit.} The filtrations on $\tEG$ and $EG$ combine to produce a filtration on $\tEG\wedge F(EG_+,
X)$.\footnote{One unpleasant aspect of this construction is that, though these filtrations are both cellular, and
the maps in~\eqref{pretate} can be made cellular, it's extremely hard to see the multiplicative structure if you do
this. This is addressed in~\cite{BMTate}.} Namely, let $\set{E_r}$ be a CW
filtration of $EG$ and $\set{\widetilde E_r}$ be a CW filtration of $\tEG$. Let
\begin{align*}
	X_{r,s} &\coloneqq \widetilde E_r\wedge F(E/E_{-s-1}, X)\\
	\overline X_{r,s} &\coloneqq \hocolim_{\substack{0\le x\le r\\0\le y\le s}} X_{x,y}\\
	\overline X_t &\coloneqq \bigcup_{r+s=t} \overline X_{r,s}.
\end{align*}
\begin{defn}
\label{tateSS}
The \term{Tate spectral sequence} is the spectral sequence induced from $\set{\overline X_t}$. It is a
\emph{conditionally convergent} spectral sequence
\[E_{s,t}^2 = \tH^s(G; \pi_tX)\Longrightarrow \pi_{s+t}X^{tG}.\]
\end{defn}
\begin{rem}
Conditionally convergent spectral sequences\index{conditionally convergent spectral sequence} may be somewhat
unfamiliar, and are less well behaved.  Boardman's ``Conditionally convergent spectral
sequences''~\cite{BoardmanCCSS} is a great paper separating the information of a spectral sequence into a
structural piece and a calculational piece (using homotopy limits). Conditional convergence means the structural
part exists. Boardman also proves a spectral sequence comparison theorem for conditionally convergent spectral
sequences! Hesselholt-Madsen~\cite{HM92} make heavy use of a nonconvergent spectral sequence to make calculations
about the $S^1$-Tate spectrum of the image-of-$J$ spectrum,\index{image-of-$J$ spectrum} which is a really
interesting use of nonconvergence. Their papers in general are a masterclass in thinking with spectral sequences,
and this one is particularly accessible.
\end{rem}
The Tate spectral sequence has a relationship to the homotopy fixed-point spectral sequence, which you might
expect.

There's an alternate construction of the Tate spectral sequence using the \term{Greenlees filtration} of $\tEG$:
\[E_r'\coloneqq \begin{cases}
	E_r, &r\ge 0\\
	DE_r, &r\le 0.
\end{cases}\]
(Here, $DE_r = \Map(E_r, \Sph)$.) Then, you can smash using $E_r'$ instead of $E_r$, and this was Greenlees'
original construction of the Tate spectral sequence.

Let's analyze the $E^1$ term a bit more. There's an isomorphism
\[\overline X_r/\overline X_{r-1}\cong\bigvee_s W_{r,s},\]
where
\[W_{r,s}\coloneqq (\widetilde E_r/\widetilde E_{r-1})\wedge F(E_{-s}/E_{-s-1}, X).\]
Now, taking $G$-fixed points,
\[\pi_{r+s+t}((\widetilde E_r/\widetilde E_{r-1})\wedge F(E_{-s}/E_{-s-1}, X)^G)\cong \pi_r(\widetilde
E_r/\widetilde E_{r-1})\otimes \pi_{s+t}(F(E_{-s}/E_{-s-1}, X)),\]
and
\[\pi_r(\widetilde E_s/\widetilde E_{r-1})\otimes\Hom(\pi_s(E_{-s}, E_{-s-1}), \pi_tX)\cong
H_r(\bl)\otimes\Hom(H_s(\bl), \pi_tX),\]
which is the Tate resolution.
\begin{rem}
As mentioned, there are connections between Tate spectra and chromatic phenomena. This is the subject of
Hopkins-Lurie's work on ambidexterity~\cite{Ambidexterity}, the astounding fact that $K(n)$-locally, $\mathit{BG}$
(and more generally any orbifold) is dualizable. This was first noticed computationally by Ravenel and
Wilson~\cite{RavenelWilson}, but can be understood as the $K(n)$-local Tate spectrum vanishing.

There are also interesting connections to the stable module category and modular representation theory.
\end{rem}
